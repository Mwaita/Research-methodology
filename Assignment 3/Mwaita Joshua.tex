\documentclass[a4paper,12pt]{article}
\begin{document}


\begin{Huge}
\begin{center}
\begin{normalsize}

\textbf{MAKERERE UNIVERSITY } \\
\textbf{FACULTY OF COMPUTING AND INFORMATICS TECHNOLOGY} \\
\textbf{DEPARTMENT OF COMPUTER SCIENCE} \\
\textbf{BACHELOR OF SCIENCE IN COMPUTER SCIENCE} \\
\textbf{BIT 2207 RESEARCH METHODOLOGY} \\
\textbf{YEAR 2} \\


\textbf{\sc MWAITA JOSHUA } \\
\textbf{\sc Reg No: 16/U/7890/PS } \\
\textbf{\sc std No: 216018350}\\
\end{normalsize}
\end{center}
\end{Huge}
\newpage

\title{HIGH INCREASE OF MALARIA PATIENTS IN KAMPALA}
\maketitle    
\section{Introduction}
Malaria is the most dangerous killer disease in Uganda. 70,000-110,000 people die of malaria every year. It has been calculated that a Ugandan person loses 23.4 percent of the years of their life to malaria.  
\section{Problem statement}                                                                                                                                                                                    
The high increase in malaria patients has been caused by the ineffective plan that government has employed to provide the Malaria preventive means to natives in the various parts of Kampala. The population in Kampala is also high as compared to the available resources .
\section{Data collection form}
The purpose of this form is to collect data that will be used to investigate on the accountability of the provision of Malaria preventive materials such as drugs and Mosquito nets supplied to the people in the sub counties of Kampala District.
The form is to collect the following types of data:
Quantitative data;
 the form will record different number of items such as number of people,Number of health centres.
 Qualitative;
 the form also should record the qualitative data such as people's view on the spread of malaria.
The following are the contents of the research form :-
 
\subsection{Name of Sub county:}
This is a text field that is used to collect the name of a given sub county in kampala.
\subsection{Number of people in a the sub county:}
This field is of type number and it collects the number of people in a given sub county in kampala

\subsection{Number of people infected with Malaria:}
The number of patients suffering from Malaria are to be recorded in this field.
\subsection{Number of reliable Health centre:}
This is used to record the number of reliable Health centres in each sub county.
\subsection{Number of Nets supplied:}
This field is to record the number of Mosquito nets supplied each year.
\subsection{Name of the health centres:}
The names of the health center to be visited will be recorded here.
\subsection{Name of the health worker being interviewed:}
The names of the health worker being interviewed are recorded in this field.
\subsection{Image of the Health centre.}
A picture of the health centres visited will be taken and uploaded to the server using this field.
\subsection{Recording of the interview.}
Every interview is recorded using this part of the form and later uploaded to the server using this field.

\subsection{Time of interview.}
The time of each interview is also recorded in this field.

Below is a table showing a sample of data that was collected:

\begin{center}
\begin{tabular}{|c|c|}
\hline
Field & Data collected  \\ [0.5ex]
\hline
Name of sub county:  & Nakawa   \\ [0.5ex]
\hline
Number of people in a sub county & 246,781, people   \\ [0.5ex]
\hline
Number of people infected with Malaria in a sub county & 70,000 people \\ [0.5ex]
\hline
Number of reliable Health centres & 20  \\ [0.5ex]
\hline
Number of Mosquito Nets supplied &500  \\ [0.5ex]
\hline
\end{tabular}
\end{center}`

\end{document}