\documentclass[12pt]{article}
\usepackage{zed-csp}
\usepackage[top=2.5cm, bottom=2.5cm, left=3cm, right=3cm]{geometry}
\usepackage{graphicx}
\begin{document}

\begin{Huge}
\begin{center}
\begin{normalsize}
\textbf{MAKERERE \includegraphics[scale=0.5]{logo} UNIVERSITY }\\


\textbf{FACULTY OF COMPUTING AND INFORMATICS TECHNOLOGY} \\
\textbf{SCHOOL OF COMPUTING AND INFORMATICS TECHNOLOGY} \\
\textbf{DEPARTMENT OF COMPUTER SCIENCE} \\
\textbf{BACHELOR OF SCIENCE IN COMPUTER SCIENCE} \\
\textbf{YEAR 2} \\
\textbf{BIT 2207 RESEARCH METHODOLOGY} \\
\textbf{Course Work: Assignment 1}\\
\end{normalsize}
\end{center}
\end{Huge}

\begin{center}
\begin{tabular}{|l|l|l|c|}
\hline NAME  & REG NO & STD NO \\\hline
MWAITA Joshua& 16/U/7890/PS & 216018350 \\\hline

\end{tabular}
\paragraph{•}
Lecturer: ERNEST MWEBAZE \\
\paragraph{•}
7th January 2018

\end{center}

\newpage


\title{•}\textbf{USING MOBILE APPLICATION TO ASSIST IN THE CONTROL OF THE RAMPANT DEATH OF CITIZENS IN UGANDA}
\section{Introduction}

\paragraph{•}There has been a massive increase in the death of citizens in Uganda over the past five years and there is every indication that this will continue. It has been noticed that many of these death are as result   of diseases and robber attacks .The government has tried to apply several techniques that have not been able to entirely overcome this problem in the country



\section{ problem Statement}
\paragraph{•}Uganda is faced with a problem of unexpected increasing death rates rising many of which are rising from diseases and others from different aspects such as robbery attacks. The government has over the past years applied methods like immunization, free health service and supply health protective materials such as mosquito nets to help in the control of these death rates. How these methods have not entirely covered the nation.

\subsection{General objectives }
\paragraph{•}The main objective of the study was to obtain information on the rampant increase in the death rates in Uganda and to derive a computational solution for this problem
\subsection{Specific objectives }
\paragraph{•}The following were the specific objectives of the study.
\paragraph{•} 1.	To identify the major cause of the death of citizens in Uganda 
\paragraph{•}2.	To gather information about the death of Ugandans so to ease the planning of the control of these diseases
\paragraph{•}3.	Estimate percentage of the different age groups affected by death rates
\paragraph{•}4.	To determine which parts of Uganda are affected by death rate
\paragraph{•}The following have been recorded as the major causes of the rampant death rates in Uganda:-

\section{DISEASE}
\paragraph{•}Over the past years diseases have killed more citizens in Uganda than any other means. Many of these diseases are air bone disease and others are spread by viruses. Some of the factors that have increased the spread of these diseases include:-
\paragraph{•}1.	Ignorance of the citizens to the means of control of the diseases
\paragraph{•}2.	Poor health facilities in the country
\paragraph{•}3.	Some of the Citizens have superstitions that cause them to refuse to take modern drugs.
\paragraph{•}4.	Inadequate funds to support the control of the diseases
\subsection{Age groups killed by diseases}
\paragraph{•}It has been noticed that children of ages from 10 years to 15 years are affected more by Malaria than adults. While adults suffer more from sexually transmitted diseases such as AIDS.
\subsection{Areas of the country affected disease outbreak}
\paragraph{•}It has been noticed that children of ages from 10 years to 15 years are affected more by Malaria than adults. While adults suffer more from sexually transmitted diseases such as AIDS.

\section{MATERNAL DEATH}
\paragraph{•}Many mothers die more during the giving of birth in hospital   and this involves mainly the young mothers who are not fully matured .this death usually affects more urban centers than the rural areas
\section{DRUG ABUSE}
\paragraph{}Drugs abuse has mostly affected more youths in the ages of 18 years to 35 years . Many youth have involved in the abusing drugs. Such drugs include marijuana, smoking and alcohol. This drug abuse has been recorded in the entire country. There more males involved in drug abuse than the females.


\section{CONCLUSION}

\paragraph{•}There has to be a way of keeping track of the information about the current and the previous death rates so as to plan for the preventive methods such:
\paragraph{•}provision of sufficient drugs in hospitals
\paragraph{•}deployment of enough health workers in the areas affected by more diseases
\paragraph{•}Decide on ways of sensitizing communities about preventive methods.

\section{RECOMMENDATIONS}
\paragraph{•}I would recommend the government to employ a reliable automated system in all government owned facilities such as hospitals and police stations to keep track on the necessary deaths in the country.
\end{document}