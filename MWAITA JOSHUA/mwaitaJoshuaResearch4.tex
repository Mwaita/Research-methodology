\documentclass[7pt]{article}
\raggedright
\parindent=0in \parskip=8pt
\usepackage{graphicx}
\usepackage[margin=1in]{geometry} % 1 inch margins all around
\begin{document}

\begin{Huge}
\begin{center}
\begin{normalsize}
\textbf{MAKERERE \includegraphics[scale=0.5]{logo} UNIVERSITY }\\

\textbf{FACULTY OF COMPUTING AND INFORMATICS TECHNOLOGY} \\
\textbf{SCHOOL OF COMPUTING AND INFORMATICS TECHNOLOGY} \\
\textbf{DEPARTMENT OF COMPUTER SCIENCE} \\
\textbf{BACHELOR OF SCIENCE IN COMPUTER SCIENCE} \\
\textbf{YEAR 2} \\
\textbf{BIT 2207 RESEARCH METHODOLOGY} \\
\textbf{Course Work: Assignment 4}\\
\end{normalsize}
\end{center}
\end{Huge}

\begin{center}
\begin{tabular}{l l l}
\textbf{NAME}  & \textbf{REGISTRATION NUMBER} & \textbf{STUDENT NUMBER} \\
MWAITA JOSHUA & 16/U/7890/PS & 216018350 \\
\end{tabular}

\paragraph{•}
\textbf{Lecturer}: Mr. Earnest Mwebaze
\end{center}

\newpage

\title{THE REASONS WHY GOOGLE CHROME IS MORE POPULAR COMPARED TO OTHER BROWSERS}
\author{MWAITA JOSHUA}      
\renewcommand{\today}{}

\maketitle

\section*{Literature Review}
\paragraph{•}
Google chrome is a web browser application introduced by Google at the very good point of time: software was moving to the cloud/into the SaaS model with Google leading the way (GMail, Google Calendar) both in consumer and enterprise space (with obvious exceptions like MS Office) and suddenly we all stopped using many small programs and switched to a web-app. A great example for me would be Winamp.\cite{Jeff}
\paragraph{•}
Chrome is much faster than any other browser because of its rendering engine and Javascript engine (V8). The same thing happens currently with Dolphin Browser on mobile - besides some UI-related advantages, it's simply faster than any other mobile browser (including Google Chrome), see: The Latest Dolphin Browser Beta Is Blazing Fast, Takes On Chrome with a New Engine. Note that according to recent benchmarks, IE10 is faster than Google Chrome (on Windows 7 64-bit), see: Benchmarks: IE10 vs. Google Chrome 21 vs. Firefox 15 vs. Opera 12\cite{Jeff}
\paragraph{•}
 Drawing on Firefox's experience, Chrome introduced it's own Extensions system. I actually did create a simple extensions and the development process is well-thougt and easy. Google also provided documentation and the whole extension system is quite powerful. This doesnot mean that it is a substantial advantage over Firefox but it shows that Google thoroughly analyzed what a modern browser should look like. Also, the current number of Chrome extensions show that it was relatively easy for developers to adapt and create extensions for Chrome.\cite{Jeff}
\paragraph{•}
Unlike the other browsers, which were created and released by software companies, Google applied a user-centric approach to Chrome. They’ve taken a lot of what they’ve learned about how users engage the web and made it easier to do the most common things. User studies show that most people find it a lot easier to use Chrome — the “omnibox” can be used for both web addresses and searches and the ability to pin your most-visited sites means that you can get started more quickly. \cite{Guilment}
\paragraph{•}
Finally, Google chrome being the most popular web browser, it should be made a default browser for most operating systems. This is because many people have adopted to the browser and it has proven to be beneficial to people. Google developers should add more function to this browser.  


  

\paragraph{•} 

\begin{thebibliography}{9}
\bibitem{Jeff} Jeff Nelson \textit{Quora}, 
Internet: www.quora.com/Why-is-Google-Chrome-so-successful, January 2013 [March 10, 2018].
\bibitem{Guilment} Lisa  Guilment. \textit{Fullcircle design} 
Internet: http://fullcircledesign.co/our-story/, November 2015 [March 10, 2018].

\end{thebibliography}

\end{document}